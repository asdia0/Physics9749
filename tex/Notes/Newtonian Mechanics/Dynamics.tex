\chapter{Dynamics}

\section{Newton's Laws of Motion}

\begin{law}[Newton's Laws of Motion]
    \phantom{.}
    \begin{enumerate}
        \item An object at rest will remain at rest and an object in motion will remain in motion at constant velocity in the absence of an external resultant force.
        \item The rate of change of momentum of a body is directly proportional to the resultant force acting on the body and occurs in the direction of the resultant force.
        \item If body A exerts a force on body B, then body B exerts a force of the same type that is equal in magnitude and opposite in direction on body A.
    \end{enumerate}
\end{law}

\subsection{Mass and Weight}

Newton's first law suggests that matter has a property called inertia.

\begin{definition}
    \vocab{Inertia} is the resistance to acceleration upon the application of a force.
\end{definition}

\begin{definition}
    The \vocab{mass} of a body is a measure of its inertia.
\end{definition}

Mass is a scalar quantity. Its SI unit is kilogram (kg).

\begin{definition}
    \vocab{Weight} is the force experienced by a mass in a gravitational field.
\end{definition}

Weight is a vector quantity. Its SI unit is newton (N).

The weight $W$ of a body is related to its mass $m$ and the acceleration of free fall $g$ by \[W = mg.\]

\subsection{Momentum, Force and Impulse}

\begin{definition}
    The \vocab{linear momentum} ($p$) of a body is the product of its mass $m$ and its velocity $v$: \[p = mv.\]
\end{definition}

Momentum is a vector quantity. Its SI unit is kilogram-metre per second (kg m s$^{-1}$). The direction of momentum is the same as the direction of the velocity.

In accordance with Newton's second law,

\begin{definition}
    \vocab{Force} ($F$) is the rate of change of momentum of the body acted upon. \[F = \der{p}{t}.\]
\end{definition}

Force is a vector quantity. Its SI unit is newton (N).

In the case of \emph{constant mass}, then \[F = \der{(mv)}{t} = m \der{v}{t} = ma.\]

\begin{definition}
    \vocab{Impulse} ($J$) is the product of the force $F$ and the time interval $\D t$ during which the force acts. \[J = F \D t = \int F \d t.\]
\end{definition}

Impulse is a vector quantity and has the same direction as the applied force. Its SI unit is newton-second (N s).

\begin{theorem}[Impulse-Momentum Theorem]
    Impulse is equal to the change in momentum.
\end{theorem}
\begin{proof}
    By Newton's second law, we have \[J = \int F \d t = \int \der{p}{t} \d t = \int \d p = \D p.\]
\end{proof}

\subsection{Action-Reaction Pairs}

Newton's third law implies that forces occur in action-reaction pairs.

\begin{definition}
    An \vocab{action-reaction pair} of forces are two forces which fulfil the following conditions:
    \begin{itemize}
        \item the two forces act on two different bodies,
        \item the forces are equal in magnitude,
        \item the forces are opposite in direction, and
        \item the forces are of the same type.
    \end{itemize} 
\end{definition}

\section{Conservation of Momentum}

\begin{law}[Conservation of Momentum]
    The total momentum of a system is constant, provided no resultant external force acts on the system.
\end{law}
\begin{proof}[Justification]
    Let body A have mass $m_1$ and initial velocity $u_1$. Let body B have mass $m_2$ and initial velocity $u_2$. Suppose bodies A and B collide. During the collision, body A exerts a force $F_2$ on body B, and body B exerts a force $F_1$ on body A. By Newton's third law, these forces are equal and opposite: \[F_2 = -F_1.\] Both forces act for the same time $\D t$, hence the impulses of the two forces are equal and opposite: \[J_2 = F_2 \D t = -F_1 \D t = -J_1.\] Since $J = \D p = m \D v$, we have \[m_2 \bp{v_2 - u_2} = -m_1 \bp{v_1 - u_1}.\] Rearranging, we get \[m_1 u_1 + m_2 u_2 = m_1 v_1 + m_2 v_2,\] so momentum is conserved.
\end{proof}

\subsection{Elastic Collisions}

\begin{definition}
    In an \vocab{elastic collision}, total kinetic energy is conserved.
\end{definition}

\begin{proposition}
    In an elastic collision, the relative speed of approach of the two bodies is equal to the relative speed of separation.
\end{proposition}
\begin{proof}
    Consider two bodies A and B, with mass $m_1$ and $m_2$ respectively, colliding head-on elastically with each other. The initial speeds of A and B are $u_1$ and $u_2$ respectively. After the collision, A and B move off with final speeds $v_1$ and $v_2$ respectively.

    By the law of conservation of momentum, \[m_1 u_1 + m_2 u_2 = m_1 v_1 + m_2 v_2.\] Rewriting, we see that
    \begin{equation}\label{eqn:elastic-1}
        m_1 \bp{u_1 - v_1} = m_2 \bp{v_2 - u_2}.
    \end{equation}
    Since kinetic energy is conserved, \[\frac12 m_1 u_1^2 + \frac12 m_2 u_2^2 = \frac12 m_1 v_1^2 + \frac12 m_2 v_2^2.\] Rewriting,
    \begin{equation}\label{eqn:elastic-2}
        m_1\bp{u_1 + v_1}\bp{u_1 - v_1} = m_2 \bp{v_2 + u_2}\bp{v_2 - u_2}.
    \end{equation}
    Dividing (\ref{eqn:elastic-2}) by (\ref{eqn:elastic-1}), we obtain \[u_1 + v_1 = v_2 + u_2 \implies u_1 - u_2 = -\bp{v_1 - v_2},\] so the relative speed of approach is equal to the relative speed of separation as desired.
\end{proof}

\subsection{Inelastic Collisions}

\begin{definition}
    In an \vocab{elastic collision}, total kinetic energy is not conserved. In a \vocab{completely inelastic collision}, the colliding bodies stick together and move off as one body after the collision.
\end{definition}